\documentclass[12pt]{article}

% Packages for formatting
\usepackage[utf8]{inputenc} % UTF-8 encoding
\usepackage[T1]{fontenc}    % Better font encoding
\usepackage{lmodern}        % Improved font rendering
\usepackage{geometry}       % Adjust margins
\geometry{
    a4paper,
    left=1in,
    right=1in,
    top=1in,
    bottom=1in
}
\usepackage{setspace}       % For line spacing
\onehalfspacing             % 1.5 line spacing for better readability
\usepackage{amsmath, amssymb, amsthm} % Math formatting
\usepackage{enumitem}       % Better lists
\usepackage{xcolor}         % Colored text for highlights
\usepackage{fancyhdr}       % Header and footer customization
\usepackage{titlesec}       % Section title formatting
\usepackage{lastpage}       % Reference last page for footer
\usepackage{float}
\usepackage{caption}

% Header and Footer
\pagestyle{fancy}
\fancyhf{}
\fancyhead[L]{Linear Algebra Final Exam}   % Left header
\fancyhead[R]{NetID: }    % Right header
\fancyfoot[C]{Page \thepage\ of \pageref{LastPage}} % Footer with page number

% Section formatting
\titleformat{\section}
  {\normalfont\Large\bfseries}{\thesection.}{0.5em}{}
\titleformat{\subsection}
  {\normalfont\large\bfseries}{\thesubsection.}{0.5em}{}

% Adjust itemize/enumerate spacing
\setlist[itemize]{itemsep=0.5em, topsep=0.5em}
\setlist[enumerate]{itemsep=0.5em, topsep=0.5em}

% Custom commands for clarity
\newcommand{\qspace}{\vspace{1em}} % Custom space between questions
\usepackage{graphicx} % Required for inserting images
\usepackage{amsmath, amssymb, amsfonts}


\title{
    \vspace{4cm}
    \textbf{\LARGE Duke University} \\
    \vspace{0.5cm}
    \textbf{\Large MATH 221: Linear Algebra and Applications} \\
    \vspace{0.5cm}
    \textbf{Fall 2023} \\
    \vspace{0.5cm}
    \textbf{\Large Final Exam} \\
    \vspace{0.5cm}
}

  
\date{}

\begin{document}

\maketitle

\textbf{Instructions:} You have \textbf{3 hours} to complete this exam, which consists of \textbf{24 questions} totaling \textbf{200 points}. Accuracy is encouraged over attempting every question. Show all your work for full credit. 

\textit{Time Estimations}

\textit{1 Hour in: At Question 10}

\textit{2 Hours in: At Question 16}

\newpage


\begin{enumerate}

    \item (12 points total)
    \begin{enumerate}
    \item (3 points) Give an example of a transformation (or map) \( T: \mathbb{R}^n \to \mathbb{R}^m \) (for some \( n, m \)) that is not a linear transformation. Explain.
    \item (3 points) Give an example of a linear transformation \( T: \mathbb{R}^4 \to \mathbb{R}^5 \) that is not injective or explain why one cannot exist.
    \item (3 points) Give an example of a linear transformation \( T: \mathbb{R}^4 \to \mathbb{R}^5 \) that is surjective or explain why one cannot exist.
    \item (3 points) Give an example of a non-identity linear transformation \( T: \mathbb{R}^n \to \mathbb{R}^n \) for some $n$ with the property $T(T(v)) = T(v)$ for all $v \in \mathbb{R}^n$.
    \end{enumerate}
    
    \item (18 points total) For the following statements, provide a proof if true or a counterexample or counterproof if false.
    \begin{enumerate}
        \item (3 points) True/False: The determinant of a triangular matrix is the product of its diagonal entries.

        \item (3 points) True/False: The set of all vectors in \( \mathbb{R}^3 \) with integer coordinates forms a subspace of \( \mathbb{R}^3 \).
        
        \item (3 points) True/False: The intersection of two subspaces of a vector space is always a subspace.
        
        \item (3 points) True/False: If \( U \) and \( V \) are subspaces of a vector space \( W \), then \( U + V \) is the smallest subspace containing both \( U \) and \( V \).

        \item (3 points) Any square matrix \( A \) can be decomposed as \( A = LL^T \), where \( L \) is lower triangular.



        \item (3 points) True/False: A square matrix must have at least one real eigenvalue.

    \end{enumerate}

    \item 
    \begin{enumerate}
        \item (3 points) Explain what happens if you feed the Gram-Schmidt algorithm a set of vectors, some of which are linearly dependent.
    
        \item (3 points) Justify the correctness of the Gram-Schmidt algorithm at a given iteration.
    \end{enumerate}

    
    \item (9 points)
    \begin{enumerate}
        \item (3 points) \( A \) and \( B \) are \( 3 \times 3 \) matrices with either determinants 5 and 10 respectively, or 10 and 5 respectively. What are the possible values for \( \det(2AB) \)?
    
    \item (3 points) Provide two classes of matrices that have their column space equal to their row space. These classes should have at least a few elements (e.g., idempotent matrices, not just the identity matrix).

    \item (3 points) What is the determinant of an orthogonal matrix? Why does this make sense geometrically?
    \end{enumerate}

    \item (5 points)         Let \( A \) be a symmetric \( n \times n \) matrix. This implies $A = Q \Lambda Q^T$, with $Q$ being orthogonal. Using this, we can write:
\[
A = Q \Lambda^{1/2} \Lambda^{1/2} Q^T = (Q \Lambda^{1/2})(Q \Lambda^{1/2})^T := BB^T.
\]

Is this factorization \( A = BB^T \) unique? If not, describe how other choices of \( B \) can satisfy \( A = BB^T \).
    
    \item (16 points total) Prove or give a counterexample.
    \begin{enumerate}
        \item (4 points) If \( B \) is similar to \( A \), then \( B^T \) is similar to \( A^T \).
        \item (4 points) If \( B \) is similar to \( A \), and \( A \) is symmetric, then \( B \) is symmetric.
        \item (4 points) If \( B \) is similar to \( A \), then \( \det(B) = \det(A) \).
         \item (4 points) Determine the determinant of this matrix by applying a similarity transform. State the similarity transformation used and justify your answer.
\begin{figure}[H]
    \centering
    \includegraphics[width=0.5\linewidth]{image.png}
    \captionsetup{labelformat=empty}
    \caption{}
\end{figure}
        \end{enumerate}
        
    
    \item (4 points) Prove that $\text{rank}(AB) \leq \min(\text{rank}(A), \text{rank}(B))$ where $A \in \mathbb{R}^{m \times n}$ and $B \in \mathbb{R}^{n \times p}$.

    
    \item (8 points) 
    Suppose that \( A \) and \( B \) are \( n \times n \) matrices and that the product \( AB \) is nonsingular. Prove that \( B \) is nonsingular and \( A \) is nonsingular in two different ways.
    \begin{enumerate}
        \item (4 points) Prove this using inverses.
        \item (4 points) Prove this using the determinant.
        
    \end{enumerate}

    \item \begin{enumerate}
        \item (4 points) Let $\{\mathbf{v}_1, \mathbf{v}_2, \dots, \mathbf{v}_n\}$ be a linearly independent set in a vector space $V$. Prove that for any $\mathbf{w} \in \text{span}(\{\mathbf{v}_1, \mathbf{v}_2, \dots, \mathbf{v}_n\})$, the scalars $c_1, c_2, \dots, c_n$ such that $\mathbf{w} = c_1 \mathbf{v}_1 + c_2 \mathbf{v}_2 + \cdots + c_n \mathbf{v}_n$ are unique.
        \item (4 points) Suppose the previous set of vectors was orthonormal. Show for any inner product, $
c_i = \langle \mathbf{w}, \mathbf{v}_i \rangle$.

        \item (4 points) Suppose the orthonormal basis was $\{ v_1, v_2, v_3\}$. Determine the possible values for $c_1, c_2, c_3$ under the following assumptions: 
        \begin{itemize}
    \item $\langle w, v_2 \rangle = 3$,
    \item $w \perp v_3$,
    \item $\lVert w \rVert = 5$.
\end{itemize}
    \end{enumerate}

    \item (5 points) Alice and Bob play a game taking turns putting elements in a $2024 \times 2024$ matrix. Alice goes first. At each turn, a player puts a real number in an open spot. The game ends when all the entries are filled. Alice wins precisely if the determinant of the finished matrix is nonzero. Who has a winning strategy and why?
        
    

    \item (9 points total) In this problem, we are going to view the space \( M_{n \times n} \) of \( n \times n \) real matrices as a vector space (of dimension \( n^2 \)).
    \begin{enumerate}
        \item (3 points) Show that the function \( T: M_{n \times n} \to M_{n \times n} \) given by \( T(X) = X^T \) is a linear transformation.
        \item (6 points) Find the eigenvalues and corresponding eigenspaces of \( T \). Give the dimensions of the eigenspaces. (Hint: show that \( T^2 \) is the identity on \( M_{n \times n} \). What does this say about the possible eigenvalues of \( T \)?)
    \end{enumerate}
    
    \item (6 points) Let \( A \) be an \( n \times n \) matrix. Consider the operator \( T(X) = AX + XA^\top \) on the space \( M_{n \times n} \) of \( n \times n \) matrices. You may use without proof that it is a linear transformation.
    
    \
    Find the eigenvalues and eigenvectors of \( T \) when \( A \) is a diagonal matrix with entries \( \lambda_1, \dots, \lambda_n \).

\item (6 points) Let \( A \) be a \( 3 \times 3 \) matrix with real entries such that \( A^2 = I \), where \( I \) is the identity matrix.
\begin{enumerate}
    \item (3 points) Show that the eigenvalues of \( A \) must be \( \pm 1 \).
    \item (3 points) Give the possible values for \( \det(A) \).
\end{enumerate}

    
    \item (8 points) Let \( V \) and \( W \) be subspaces of \( \mathbb{R}^n \) with \( V \cap W = \{0\} \). Let \( S = \text{proj}_V \) and \( T = \text{proj}_W \). Show that \( S \circ T = T \circ S \) if and only if \( V \) and \( W \) are orthogonal (i.e., \( v \cdot w = 0 \) for all \( v \in V \), \( w \in W \)). 
    
    \item (12 points total) Suppose that \( N \) is a nilpotent \( n \times n \) matrix (this means \( N^r = 0 \) for some positive integer \( r \)).
    \begin{enumerate}
        \item (4 points) Show that \( 0 \) is the only eigenvalue of \( N \).
        \item (4 points) Prove $\det(N + I) = 1$. (Hint: eigenvalues)
        \item (4 points) Show that $ANA^{-1}$ is also nilpotent for the same $r$.
    \end{enumerate}

    \item (15 points total)
    \begin{enumerate}
        \item (8 points) Prove that a symmetric matrix \( A \) satisfies \( x^\top A x \geq 0 \) for all \( x \in \mathbb{R}^n \) if and only if \( A \) has only non-negative eigenvalues.
    
        \item (3 points) Let \( B \) be a real \( n \times n \) matrix. Argue why \( B^T B \) is diagonalizable and that all the eigenvalues of \( B^T B \) are nonnegative.

        \item (4 points) Prove that \( A^\top A + I \) is invertible for any matrix \( A \).
    \end{enumerate}

        
    
    \item (3 points) If \( U \) is an orthogonal matrix, argue that \( \| Ux \|_2^2 = \| x \|_2^2 \).
    
    \item (3 points) If \( x^\top A^\top A x = 0 \), what can we say about \( x \)?

        \item (6 points) The Vandermonde matrix $V$ below is used to a polynomial of degree $n-1$ that passes through the $n$ points. 

    $$\begin{pmatrix}
1 & x_1 & x_1^2 & \cdots & x_1^{n-1} \\
1 & x_2 & x_2^2 & \cdots & x_2^{n-1} \\
\vdots & \vdots & \vdots & \ddots & \vdots \\
1 & x_n & x_n^2 & \cdots & x_n^{n-1}
\end{pmatrix}
\begin{pmatrix}
a_0 \\
a_1 \\
a_2 \\
\vdots \\
a_{n-1}
\end{pmatrix}
=
\begin{pmatrix}
y_1 \\
y_2 \\
\vdots \\
y_n
\end{pmatrix}$$
    \begin{enumerate}
        \item 

Suppose I told you $\det(V) = \prod_{1 \leq i  < j \leq n} (x_j - x_i)$. Prove $V$ is invertible if and only if the $x_i$ are distinct.

\item Explain why invertibility is useful in this application and how it relates to the uniqueness of polynomials that fit the points.
    \end{enumerate}
    
    
    \item (8 points) Suppose that \( A \) and \( B \) are real symmetric matrices that commute (i.e., \( AB = BA \)). Prove that there is an orthogonal matrix \( Q \) such that both \( Q^{-1} A Q \) and \( Q^{-1} B Q \) are diagonal. (Hint: Explain why it suffices to prove that \( B \) preserves each eigenspace of \( A \). Prove this fact using the fact that \( A \) and \( B \) commute.)

     \item (12 points total) 
    \begin{enumerate}
        \item (6 points) If \( A \) is an \( n \times n \) matrix with integer entries, prove that \( A \) has an inverse with integer entries if and only if \( \det(A) = \pm 1 \).
        \item (6 points) Show that if \( A \) and \( B \) are \( 2 \times 2 \) matrices with integer entries, and \( A, A+B, A+2B, A+3B, \) and \( A+4B \) all have inverses with integer entries, then the same is true for \( A + tB \) for all integers \( t \). (Hint: consider \( \det(A + tB) \) as a polynomial in \( t \). Show that it is always equal to \( 1 \) or \( -1 \).)
    \end{enumerate}


    
    \item (6 points) Suppose that \( A \) is an \( m \times n \) matrix with rank \( m \). Suppose that \( v_1, \ldots, v_k \) span \( \mathbb{R}^n \), that is to say, for every \( x \) in \( \mathbb{R}^n \) there exist real coefficients \( c_1, \ldots, c_k \) such that \( x = c_1 v_1 + \ldots + c_k v_k \). Show that \( \{ A v_1, \ldots, A v_k \} \) span \( \mathbb{R}^m \).

    
    \item (5 points ) Let \( A \) be the matrix \( \begin{pmatrix} 1 & 2 & 3 \\ 4 & 5 & 6 \end{pmatrix} \).
      
      Prove or disprove: For all \( b \) in \( \mathbb{R}^2 \), there exists \( u_0 \) in \( \mathbb{R}^3 \) such that \( Au_0 = b \), and so that the following sets \( U \) and \( V \) are equal.
        \[
        U = \{ u : Au = b \}
        \]
        \[
        V = \{ u : u = u_0 + v, \text{ where } Av = 0 \}
        \]


    \item (6 points) Let \( A \) be any \( n \times n \) matrix with real entries, and let \( I_n \) denote the \( n \times n \) identity matrix. Show that
    \[
    \det(I_n + A^2) \geq 0.
    \]
    Note: this is a generalization of $I_n+A^TA$ except it can now be singular. 
    \item (Bonus Questions - Extra Credit) Let \( A = (a_{ij}) \) be a real \( n \times n \) matrix satisfying
\[
|a_{ii}| > \sum_{j \neq i} |a_{ij}|
\]
for all \( 1 \leq i \leq n \). Prove that \( A \) is invertible.


\noindent\text{Note:} These questions are tricky, and there are easier ways to earn points on this exam. No partial credit will be awarded.    
    
\end{enumerate}

\end{document}
